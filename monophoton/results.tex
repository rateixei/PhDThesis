\section{Results}

    To evaluate the 95$\%$ confidence level (CL) limits on the new physics production cross section, an asymptotic CL$_{S}$ method~\cite{cls,cls1} is used where the systematic uncertainties in the signal and background predictions are treated as nuisance parameters with log-normal prior distributions. %Nuisances arising from uncertainties are applied for the normalization of signal and background processes. 

%%---------Model Independent limits
\subsection{Model-independent limits}

  Due to the variety of signals which can contribute to this final state, we present results for a generic signal using the model-independent selection described in Section~\ref{sec:event_selection}. Although this selection does not have as strong discrimination power between signal-like and background-like events compared to the misreconstructed \met rejection selections, it does have less model dependence. This is due to \met significance and $\tilde{\met}$ minimization requirements having a non-trivial efficiency dependence on the underlying event and observed \met.

  The total expected SM background and observed data events after the model-independent selection are found to be compatible within the systematic uncertainties. Table~\ref{table:modelInd} shows a comparison of the event yields estimated for background processes and the observed data. Figure~\ref{fig:modelInd} shows the $M_{T}$ and \met distributions after the model-independent selection has been applied.


\begin{table}[H]                                                                  
\center    
{          
\begin{tabular}{|c|c|}                                                             
\hline     
Process & \# of Events \\                                                              
\hline     
$\gamma +$ jets                          & $(313 \pm 50 ) \times 10^3$ \\
${\rm jet}\rightarrow \gamma$        & $(906 \pm 317 ) \times 10^2$ \\
${\rm e} \rightarrow \gamma$         & $(1035 \pm 62 ) \times 10^1$ \\
$W(\to \ell\nu)+\gamma $                 &  $2239 \pm 111$ \\
$Z( \to \nu \bar{\nu} )+\gamma    $      &  $2050 \pm 102$ \\
Other                                    &  $1809 \pm 91$ \\
\hline
Total background                       &   $(420 \pm 82 ) \times 10^3$ \\
\hline
Data                                   &  $442 \times 10^3$  \\
\hline     
\end{tabular}  
\caption{Comparison of event yields for observed data and background, after the model-independent selection.}
\label{table:modelInd}
}          
\end{table}

  %% figure of ModInd spectrun selection 
\begin{figure}[H]                  
\centering   
%{\label{fig:QCDPt}\includegraphics[scale=0.4]{Unblinding/ModelIndep/StackedHisto_MT.pdf}}                           
%{\label{fig:QCDMET}\includegraphics[scale=0.4]{Unblinding/ModelIndep/StackedHisto_MET.pdf}}                         
{\label{fig:QCDPt}\includegraphics[width=0.45\textwidth]{PAS_Plots2/StackedHisto_MT.pdf}}                           
{\label{fig:QCDMET}\includegraphics[width=0.45\textwidth]{PAS_Plots2/StackedHisto_MET.pdf}}                         
\caption{The $M_{T}$ and \met distributions for data, background estimates, and signal after the model-independent selection. The bottom panels in each plot show the ratio of (data - background)/background and the gray band includes both the statistical and systematic uncertainty on the background prediction. }                           
\label{fig:modelInd}                     
\end{figure} 

  Figure~\ref{fig:limit_MI} shows the observed and expected model-independent $95\%$ CL upper limits on $pp\rightarrow h$ cross section times the exotic Higgs decay branching ratio times the acceptance and efficiency of selecting the signal ($\sigma\times BR\times A\times\epsilon$) for different \met and $M_{T}$ thresholds (typical values for the signal efficiency $\epsilon$, in the case where $M_{\chi} = 120$ GeV, are shown in Table \ref{tab:cuts}). The observed and expected limits are also shown in Fig~\ref{fig:limit_MI}(c) at a 95$\%$ CL for $M_{T}> 100$ GeV and as a function of \met.


\begin{figure}[H]        
\centering                
{\includegraphics[width=0.45\textwidth]{PAS_Plots/pretty/2d_expected.pdf}}
{\includegraphics[width=0.45\textwidth]{PAS_Plots/pretty/2d_observed.pdf}} \\
{\includegraphics[width=0.45\textwidth]{PAS_Plots/LimitPlotVsMET_MT100.pdf}}
%{\includegraphics[scale=0.4]{Unblinding/ModelIndep/2d_final_expected.pdf}}
%{\includegraphics[scale=0.4]{Unblinding/ModelIndep/2d_final_observed.pdf}} \\
%{\includegraphics[scale=0.4]{Unblinding/ModelIndep/LimitPlotVsMET_MT100.pdf}}
\caption{ The expected (top left) and observed (top right) 95$\%$ CL upper limit on $\sigma \times BR\times A\times\epsilon$ for different $M_{T}$ and \met thresholds and (bottom) for $M_{T}> 100$ GeV as function of the \met threshold.}
\label{fig:limit_MI}      
\end{figure}    


\subsection{Model-specific limits}

   %%-------Higgs EXO limit 
   The yields for supersymmetric decays of the Higgs boson ($h\to \PXXSG\PSGczDo,\PSGczDo\to\PXXSG\gamma$) are calculated through imposing the model-specific selection described in Section~\ref{sec:event_selection}. The yields for this selection are shown in Table~\ref{tab:exoh}. The 95$\%$ CL upper limits on the $\sigma \times$ branching ratio(BR) and ($\sigma \times BR )/ \sigma_{SM}$, where $\sigma_{SM}$ is the cross section for the standard model Higgs boson, are evaluated for different mass values of \PSGczDo ranging from 65\GeV to 120\GeV and are shown in Fig.~\ref{fig:limit_higgs}.   
                       
\begin{table}[H]
\centering
\begin{tabular}{|c|c|}
\hline
Process 						& 			Estimate \\ \hline
$\gamma +$ jets                         	& 			179 $\pm$ 28 \\
${\rm jet}\rightarrow \gamma$		& 			269 $\pm$ 94 \\
${\rm e} \rightarrow \gamma$		&			355 $\pm$ 28 \\
$W(\to \ell\nu)+\gamma $			&			154 $\pm$ 15 \\
$Z( \to \nu \bar{\nu} )+\gamma$  	&  			182 $\pm$ 13 \\
Other                                    		&  			91 $\pm$  10 \\ \hline
Total background                       		&   			1232 $\pm$ 188 \\ \hline
Data                                   		&  			1296  \\ \hline \hline
$M_{\PSGczDo}$ = 65~\GeV 		& 			653.0 $\pm$ 77 \\
$M_{\PSGczDo}$ = 95~\GeV 		& 			1158.1  $\pm$ 137\\
$M_{\PSGczDo}$ = 120~\GeV 		& 			2935.0 $\pm$ 349 \\ \hline
\end{tabular}

\caption{Expected (SM background) and observed event yields after the selection optimized for the supersymmetric decay of the Higgs boson ($h\to\PXXSG\PSGczDo,\PSGczDo\to\PXXSG\gamma$) and the signal predictions  correspond to BR($H \rightarrow $invisible + $\gamma$) =100\%.}

\label{tab:exoh}

\end{table}


\begin{figure}[H]
\centering
{\includegraphics[width=0.45\textwidth]{PAS_Plots2/limit_xsec.pdf}}
{\includegraphics[width=0.45\textwidth]{PAS_Plots2/limit_ratio.pdf}}
%{\includegraphics[scale=0.4]{Unblinding/Susy/ratio.pdf}}
%{\includegraphics[scale=0.4]{Unblinding/Susy/xsec.pdf}}
\caption{ Expected and observed 95\% CL upper limits on (a) $\sigma \times BR$ and (b) the ratio of this product over the SM Higgs production cross section as a function of different $M_{\PSGczDo}$ values. The uncertainty on the expected limit at 1$\sigma$ and 2$\sigma$ levels are shown as green and yellow bands, respectively. }
\label{fig:limit_higgs}    
\end{figure}


  %%-----DM Limits and plots
%   In the case of dark matter production, lower bounds are evaluated for the cut-off scale $\Lambda$ and then converted to limits on the $\chi$-nucleon cross section using the relation in Eqn.~\ref{eq:dmXS} under the EFT approximation.

 %%-----DM Table Vector
%\begin{table}[H]                                                                  
%\center    
%{ 
%\begin{tabular}{|c|c|c|c|}                                                             
%\hline
% $M_{\chi}$ (\GeV) & $\Lambda$ (\GeV) & $\sigma$ (fb) & $\chi$-nucleon ($\cm^{2}$) \\       
%\hline
% 1    &  ()  &  () & ()  \\
% 10   &  () &  () &  () \\
% 100  &  () & ()  &  () \\
% 200  &  () & ()  &  () \\
% 500  &  () & ()  &  () \\
% 1000 &  () & ()  &  () \\
%\hline
%\end{tabular}  
%\caption{The observed (expected) 90$\%$ CL upper limit on production cross section ($\sigma$), 90$\%$ CL lower bounds on scale $\Lambda$ and $90\%$ CL upper limits on $\chi$-nucleon cross section within the EFT framework for vector operator, for different $M_{\chi}$ masses.}                     
%\label{table:Vdm}                                                                 
%}
%\end{table}

 %%-----DM Table Axial-Vector
%\begin{table}[H]                                                                  
%\center    
%{ 
%\begin{tabular}{|c|c|c|c|}                                                             
%\hline
% $M_{\chi}$ (\GeV) & $\Lambda$ (\GeV) & $\sigma$ (fb) & $\chi$-nucleon ($\cm^{2}$) \\       
%\hline
% 1    & ()  & ()  & ()  \\
% 10   & ()  & ()  & ()  \\
% 100  & ()  & ()  & ()  \\
% 200  & ()  & ()  & ()  \\
% 500  & ()  & ()  & ()  \\
% 1000 & ()  & ()  & ()  \\
%\hline
%\end{tabular} 
%\caption{The observed (expected) 90$\%$ CL upper limit on production cross section ($\sigma$), 90$\%$ CL lower bounds on scale $\Lambda$ and $90\%$ CL upper limits on $\chi$-nucleon cross section within the EFT framework for axial-vector operator, for different $M_{\chi}$ masses.} 
%\label{table:AVdm}                                                                 
%}
%\end{table}

 %%------- Limit ploto for DM
%\begin{figure}[H]
%\centering
%{\includegraphics[scale=0.7]{figures/LowEt_DM.pdf}}
%%{\includegraphics[scale=0.45]{figures/exoh_limit.pdf}}
%\caption{ The 90$\%$ CL upper limits on $\chi$-nucleon cross section for (a) vector and (b) axial-vector interaction for $\gamma\chi\chi$ production under the EFT assumption. The direct detection results from different experiments~\cite{CDMS2,XENON100,PICASSO,COUPP,COGENT,CDMSLITE,LUX} are also shown for comparison. The limits from high \pt searches at $\sqrt{s}= 8$ TeV in the monophoton final state are also shown from the CMS~\cite{CMS:2014mea} and ATLAS~\cite{ATLASmono} experiments. }
%\label{fig:DMlimits}
%\end{figure}

%  Tables~\ref{table:Vdm} and \ref{table:AVdm} show the $90\%$ CL lower bound on $\Lambda$ and corresponding upper limits on the $\chi$-nucleon cross section for different $M_{\chi}$ values. These limits are evaluated for spin-independent and spin-dependent operators. Fig.~\ref{fig:DMlimits} shows these limits as a function of $M_{\chi}$ and compares these results with other direct detection experiments~\cite{CDMS2,XENON100,PICASSO,COUPP,COGENT,CDMSLITE,LUX}. The results from previous high \pt ( \pt $> 125$ GeV) searches at the LHC are also shown~\cite{CMS:2014mea,ATLASmono}.   


























