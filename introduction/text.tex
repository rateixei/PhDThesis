\chapter{Introduction}

\section{The Standard Model of Particle Physics}

The Standard Model of Particle Physics (SM) is the model for elementary particles and their interactions, built upon a quantum field theory framework, that has been observed to explain most phenomena at distances smaller than atomic scales. 
The model is based on the existence of elementary fermionic fields that interact via bosonic fields. 
These interactions are constructed via the principle of gauge symmetry, which relates local and continuous gauge symmetries on the SM lagrangean to new interacting fields. 

Three different types of interactions are currently modeled by the SM: the strong interactions, also known as quantum chromodynamics (QCD), the weak interactions and the electromagnetic interactions. 
QCD is  generated in the SM via an $SU(3)$ symmetry on a "color" charge - fermions that contain color charge are called quarks. 
Similarly, the electroweak interactions are generated in the SM via an $SU(2)$ symmetry on a "weak isospin" charge. 

\section{Higgs Physics Current Status}
\section{Higgs as a Probe to New Physics}
Einstein's paper: \cite{Einstein}
