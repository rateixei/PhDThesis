\chapter{Conclusions}

This thesis is devoted to the study of the Higgs boson as a portal to beyond the Standard Model physics, a theoretically and experimentally rich pursuit at the LHC experiments.  
The capability of coupling the Higgs boson to new physics ensures the Higgs as an invaluable tool in this new era of particle physics.

The analyses presented in this document look for new physics coupling with the Higgs both in its decay (Higgs decays with photons and MET) and in its production (Higgs pair production).
The Higgs decays with photons and MET analysis at CMS was the first of its kind in this exotic final state, but also in exploring a new kinematic regime for photons and MET production. 
The di-Higgs search in the $b\bar{b}\gamma\gamma$ final state, with CMS Run 2 data, was a continuation of a long effort, dating back to Run 1, which culminated in the strongest limit on SM-like HH production to date (as of April 2017). 
While built on solid ground, many new features were added to the 2016 $HH\rightarrow b\bar{b}\gamma\gamma$ analysis, such as the novel categorization scheme and the dedicated b-jet energy regression. These changes improved the analysis sensitivity to SM-like HH production by over $30\%$ from the 2015 analysis baseline.

Both analyses presented in this thesis profit enormously from the performance, online and offline, of the CMS electromagnetic calorimeter. 
A large part of the work performed during the research period of this thesis was dedicated to ensuring this performance, particularly online, as a part of the ECAL DAQ team responsible for maintaining and upgrading the online ECAL software. 
This included taking leadership roles during the Run 2 recommissioning period in re-integrating ECAL in the upgraded CMS central DAQ framework and interfacing with the newly installed TCDS. 
From 2014 to 2016, several DAQ tools were also integrated in the ECAL online software, such as back-end electronics monitoring tools and the ability to perform different and independent types of runs. 

Even though it's been running for almost a decade now, the LHC program is still in its infancy. 
Given the programmed upgrades beyond 2020, with 3000 fb$^{-1}$ expected to be delivered in total, we are still at about $1\%$ of the overall amount of data to be analyzed by the LHC experiments. 
The full LHC dataset will provide an invaluable opportunity to study the SM at high energies with an unparalleled  precision. 
Even the most remote corners will be available then, such as the elusive Higgs self-coupling, explored in this thesis with the LHC Run 2 dataset. 
While physics beyond the Standard Model remains elusive, the search goes on - it might just be right around the next corner.

 