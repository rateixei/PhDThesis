\begin{acknowledgementslong}

First and foremost, I would like to thank my Ph.D. advisor, Prof. Toyoko Orimoto, for her encouragement and support, particularly at tough times. 
After spending almost four years based at CERN hearing horror stories about life-draining and mind-destroying Ph.D. advisors, I consider myself extremely lucky to have worked with someone who would push me into new projects and adventures while still considering my well-being. 

I would like to thank the second in command, one of the most upbeat people I have ever met, Italian kitchen master and ECAL connoisseur Andrea Massironi. 
A good part of what I have learned during my years at CERN were from white-board conversations with him and/or at pizza parties. 

I would like to thank the ECAL DAQ team, particularly Giacomo Cucciati. Learning the ropes of the ECAL online software while cheering in the office whenever the code compiled was surely a great way to work.

I would like to thank, in general, the ECAL team, and, specifically, David Barney, Andr� David, Evgueni Vlassov, David Petyt and Pedro Parracho. Their expertise is only surpassed by their willingness to teach newcomers about the wonders of ECAL.

I would like to thank the $b\bar{b}\gamma\gamma$ team, specially Amina Zghiche and Alexandra Carvalho, for the support and encouragement even when the energies were gone but there was still a lot to do. 

I would also like to thank the Northeastern HEP team, specially Profs. Darien Wood and Emanuela Barberis, for their continuous useful input and discussions during our group meetings. And to Bruno, Dina, Mark and Mitchell, who made my one year in Boston an unforgettable experience.

A special warm thanks to Rafael Coelho Lopes de S�, who followed me from D0 to CMS so he could continue teaching me about particle physics and the wonders of the academic life. His help during the always-so-difficult job-searching period probably kept me from throwing my laptop against a wall a couple of times. 

Brenda, Brisa, Rebeca and Yasmin, you know the drill. 

Finalmente, agrade�o a minha m�e, Roseane Teixeira de Lima, que me apoia mesmo se eu deixar de ligar para ela na hora em que sempre combinamos - ela sabe que n�o fa�o por mal. 
Mas agrade�o especialmente a ela pela minha forma��o como uma pessoa de sonhos e objetivos concretos, e que trabalha por eles. 
A minha m�e foi o meu primeiro exemplo de que as coisas melhoram com trabalho duro e dedica��o. 
As coisas melhoraram tanto que ela j� conheceu v�rios pa�ses da Europa e vai voltar para mais.

\end{acknowledgementslong}