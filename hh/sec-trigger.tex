\section{Analysis objects and selection}
This analysis uses general purpose reconstruction of photons and jets, which have been described in previous sections. 
We have brief descriptions below, pointing out specific choices made for the analysis.

\subsection{Triggers And Pre-Selection}
\label{sec:trigger}
Exploiting the high online performance of the CMS ECAL to reconstruct photons and electrons, the dataset used in this analysis is constructed with a selection that requires two photons at High Level Trigger (HLT) level.

For the 2016 data taking period, the online strategy was based on a single HLT trigger path, based on selecting two photons, one with $E_T > 30$ GeV (leading) and one with $E_T > 18$ GeV (subleading). Quality criteria are required on these photons, such as on calorimeter isolation (the amount of calorimeter activity around the reconstructed photon), on the ratio between the energies deposited on ECAL and HCAL, and on R9 (the ratio between the energy deposited on a 3x3 ECAL crystal matrix around the most energetic crystal in the supercluster, and the supercluster energy). 
Additionally, it is required, at trigger level, that $M(\gamma\gamma)>90$ GeV. 

In order to achieve good data/simulation comparison, a pre-selection that is tighter than the online selection is applied on data and Monte Carlo. This pre-selection is described in Table \ref{tab:preselection}. 
It is based on shower shape variables ($R9$), isolation variables (charged hadron isolation, CHI, the sum of all charged hadron particle flow candidates energies inside a cone of $\Delta R < 0.3$ around the photon axis), identification variables (H/E, the ratio between the photon's energy deposit in HCAL and in ECAL), and kinematic variables ($E_{T}$ and the photon supercluster $\eta$). Only events that have at least one diphoton candidate passing the pre-selection requirements are considered in the analysis.

 \begin{table}[h]
\centering
\small{
\begin{tabular}{rcc}
Requirements & Leading Photon & Subleading Photon \\ \hline
$E_{T}$ & $30 \gev$ & $20 \gev$ \\ \hline
$|\eta|$ & \multicolumn{2}{c}{ $< 2.5$ and outside $1.442  |\eta| < 1.566$ } \\ \hline
Shower shape and Isolation & \multicolumn{2}{c}{ $R9 > 0.8$ or CHI $< 20$ or CHI/$E_{T} < 0.3$} \\ \hline
Identification & \multicolumn{2}{c}{ H/E $< 0.08$} \\ \hline
\end{tabular}
}
\caption{\small Trigger based pre-selection applied on diphoton candidates.  \label{tab:preselection}}
\end{table}

The SM $\Hgg$ analysis provides scale factors and uncertainties related to this selection, which we also apply in the analysis and include in our systematical uncertainties. 
