\section{MVA Based Categorization}
\label{sec:cats}

During the analysis, it has been noticed that different kinematic variables could potentially contribute to constraining the background contribution in the signal region without cutting too much on the signal efficiency. 
However, this large-dimensional optimization procedure (all investigated variables) was not optimal. 
Instead, we have developed a multivariate analysis (MVA), combining these different variables, into a single discriminant. 
This discriminant is used to categorize the events in High Purity, Medium Purity categories and a control region, similarly to the cut based categorization. 

The input variables investigated for this MVA were:
\begin{itemize}
\item Leading and subleading jets b-tagging score;
\item Helicity angles $|cos(\theta^{*}_{CS})|$, $|cos(\theta^{*}_{bb})|$ and $|cos(\theta^{*}_{\gamma\gamma})|$: $|cos(\theta^{*}_{CS})|$ is defined as the angle between the direction of the $H\rightarrow\gamma\gamma$ candidate to the Colin-Sopper reference frame (assumes each incoming particle in the scattering to have 6.5 TeV); $|cos(\theta^{*}_{xx})|$ is defined as the angle between the particle $x$ and the direction defined by the $H\rightarrow x x$ candidate (randomly choosing between x's), where $x = \gamma$ or $b$;
\item $p_{T}(\gamma\gamma)/M(jj\gamma\gamma)$ and $p_{T}(jj)/M(jj\gamma\gamma)$
\end{itemize}

The training was performed in the photon control region, as described in \ref{sec:PCR}. 
Plots comparing the input variables in the photon control region and the blinded signal region are shown in Figure \ref{fig:inputmva}. 
As our signal in the training, we sum the 14 non-resonant HH samples available (box only, SM, and 12 BSM points). 
Thus, we have a training that is not specific to a single region in the parameter space, maintaining the sensitivities comparable between the benchmark points (as it is with the cut based categorization). 

\begin{figure*}[h]
  \centering
  \includegraphics[width=0.3\textwidth]{figures/sec-cats/mva/ljbdis}\hfil
  \includegraphics[width=0.3\textwidth]{figures/sec-cats/mva/sjbdis}\hfil
  \includegraphics[width=0.3\textwidth]{figures/sec-cats/mva/cts_cs}\hfil
  \includegraphics[width=0.3\textwidth]{figures/sec-cats/mva/ct_bb}\hfil
  \includegraphics[width=0.3\textwidth]{figures/sec-cats/mva/ct_gg}\hfil
  \includegraphics[width=0.3\textwidth]{figures/sec-cats/mva/gghhr}\hfil
  \includegraphics[width=0.3\textwidth]{figures/sec-cats/mva/bbhhr}\hfil
  \caption{Distributions of input variables in the blinded photon control region, blinded signal region, and SM HH sample. All normalized to unity. }
  \label{fig:inputmva}
\end{figure*}

To improve the training, we split the training into two regions: low mass and high mass. 
The low mass training is performed with events with $\tilde{M}_{X}$ below 350 GeV, while the high mass training uses the complementary region. 
The training is based on a decision tree boosted with the gradient algorithm, with the trees randomized between iterations to decrease overtraining. 
To implement the training the TMVA package was used. 
%The TMVA output plots are shown for both trainings in Figures \ref{fig:mva_hm} and \ref{fig:mva_lm}. 
From now on, we will refer to the training discriminant variable as HHTagger.

With the HHTagger discriminant, we build our two signal categories based on the optimization of the expected SM HH limit, separately in the high mass and in the low mass regions. 
These categories are called high purity category (HPC) and medium purity category (MPC); their names refer to their signal and background content. 
As signal, we use the SM HH sample to calculate the sensitivity. 
The outcome of this study was the categorization in Table \ref{tab:catmva}. 
The expected number of background events, when comparing MPC and HPC between cut based and MVA approaches, is comparable and consistent, while for the number of signal events, the performance is better for the HHTagger categorization. 

We also have to ensure that the HHTagger selection does not distort our variables of interest, $\Mjj$ and $\Mgg$. 
We demonstrate that there is no appreciable shaping by comparing the $\Mjj$ and $\Mgg$ shapes in different bins of the HHTagger discriminant. 
This can be seen in Figure \ref{fig:mva_mggmjj}. 

\begin{table}
\centering
    \begin{tabular}{| c | c | c |}
    \hline
    Mass Region & HPC & MPC \\ \hline
    Low Mass & HHTagger $> 0.96$ & $ 0.75 < $ HHTagger $ < 0.96 $ \\ \hline 
    High Mass & HHTagger $> 0.96$ & $ 0.6 < $ HHTagger $ < 0.96 $ \\ \hline 
    \end{tabular}
\caption{Non-resonant categorization with HHTagger discriminant.}
\label{tab:catmva}
\end{table}


%\begin{figure*}[h]
%  \centering
%  \includegraphics[width=0.45\textwidth]{figures/sec-cats/mva/vars1_hm400}\hfil
%  \includegraphics[width=0.45\textwidth]{figures/sec-cats/mva/vars2_hm400}\hfil
%  \includegraphics[width=0.45\textwidth]{figures/sec-cats/mva/corsS_hm400}\hfil
%  \includegraphics[width=0.45\textwidth]{figures/sec-cats/mva/corsB_hm400}\hfil
%  \includegraphics[width=0.45\textwidth]{figures/sec-cats/mva/ROC_hm400}\hfil
%  \includegraphics[width=0.45\textwidth]{figures/sec-cats/mva/discr_hm400}\hfil
%  \caption{TMVA output plots for the High Mass Training.}
%  \label{fig:mva_hm}
%\end{figure*}

%\begin{figure*}[h]
%  \centering
%  \includegraphics[width=0.45\textwidth]{figures/sec-cats/mva/vars1_lm400}\hfil
%  \includegraphics[width=0.45\textwidth]{figures/sec-cats/mva/vars2_lm400}\hfil
%  \includegraphics[width=0.45\textwidth]{figures/sec-cats/mva/corsS_lm400}\hfil
%  \includegraphics[width=0.45\textwidth]{figures/sec-cats/mva/corsB_lm400}\hfil
%  \includegraphics[width=0.45\textwidth]{figures/sec-cats/mva/ROC_lm400}\hfil
%  \includegraphics[width=0.45\textwidth]{figures/sec-cats/mva/discr_lm400}\hfil
%  \caption{TMVA output plots for the Low Mass Training.}
%  \label{fig:mva_lm}
%\end{figure*}

\begin{figure*}[h]
  \centering
  \includegraphics[width=0.45\textwidth]{figures/sec-cats/mva/hhtag_mgg}\hfil
  \includegraphics[width=0.45\textwidth]{figures/sec-cats/mva/hhtag_mjj}\hfil
  \caption{Photon control region distributions of $\Mgg$ (left) and $\Mjj$ (right) in bins of HHTagger. Although the slope changes between bins, this effect does not influence the limit setting.}
  \label{fig:mva_mggmjj}
\end{figure*}

\subsubsection{Performance Cross-Checks}

We have performed several cross checks to look for possible improvements on the MVA categorization. 

\begin{itemize}
\item \textbf{Signal Hypothesis}
\end{itemize}

In the standard training, the sum of all non-resonant samples are used as the signal hypothesis. 
However, this might not be the optimal training for the SM HH case. 
To test this, we compare the performance of different trainings assuming the SM HH signal. 
The signal hypotheses tested are:
\begin{itemize}
\item All non-resonant (standard);
\item SM HH;
\item SM HH, with separate training for the high mass and low mass region (similar to standard);
\item SM HH + Node 3 (this node, as defined in Table \ref{tab:bench_old}, contains the SM point);
\item SM HH + Node 3, with separate training for the high mass and low mass region (similar to standard).
\end{itemize}
The background hypothesis for this test is the photon control region, in the high mass region. 

\begin{figure*}[h]
  \centering
  \includegraphics[width=0.7\textwidth]{figures/sec-cats/mva/ROC}\hfil
  \caption{ROC curves with different signal hypotheses for training. The performance is evaluated in the high mass region, with the photon control region as background and SM HH as signal.}
  \label{fig:mva_cc_signal}
\end{figure*}


The ROC curves from the different trainings are shown in Figure \ref{fig:mva_cc_signal}. 
Since no significant improvement is seen in the high purity region (for background rejection larger than 95\%, a typical value for the chosen WPs), the standard training method is kept in use. 

\begin{itemize}
\item \textbf{Background Hypothesis}
\end{itemize}

In the standard training, the photon control region is used as a background model, avoiding MC reliance. 
However, this might not be optimal because the photon control region might have different correlations between the MVA variables with respect to the signal region. 
To test this, we compare the performance of different trainings assuming different background hypotheses: 
\begin{itemize}
\item Photon control region (standard);
\item Blinded signal region;
\item Blinded control region (to ensure that the difference between the two previous trainings does not come from blinding).
\end{itemize}
The background hypothesis for this test is the blinded signal region, in the high mass region, and the signal hypothesis is SM HH. 

\begin{figure*}[h]
  \centering
  \includegraphics[width=0.7\textwidth]{figures/sec-cats/mva/ROC_BCR}\hfil
  \caption{ROC curves with different background hypotheses for training. The performance is evaluated in the high mass region, with the blinded signal region as background and SM HH as signal.}
  \label{fig:mva_cc_background}
\end{figure*}

The ROC curves from the different trainings are shown in Figure \ref{fig:mva_cc_background}. 
While some improvement is seen, this training is not optimal for statistical reasons. 
The blinded signal region contains significantly less events than the photon control region. 
This limits the precision and accuracy of the multivariate analysis training. 
Specifically, it has been observed that the blinded signal region does not contain events in the high BDT region (signal-like phase space), which can cause over training. 
The second issue is that, optimally, training on a dataset that is statistically independent from the one to which it will be applied leads to a more robust procedure. 



\begin{itemize}
\item \textbf{Resonant Hypothesis}
\end{itemize}

\begin{figure*}[h]
  \centering
  \includegraphics[width=0.7\textwidth]{figures/sec-cats/mva/ROC_res}\hfil
  \caption{ROC curves with different signal hypotheses for training. The performance is evaluated in the low mass region, with the photon control region as background and the 300 GeV radion sample as signal.}
  \label{fig:mva_cc_res}
\end{figure*}

While this MVA is trained with the non-resonant signal hypotheses, it can also be applied to the resonant search. 
We check, however, if a dedicated training with the resonant samples as signal hypothesis performs better, when applying the categorization to the resonant analysis.  
This is tested by comparing the categorization performance of the standard training versus a resonant training on a resonant signal point. 
The plot in Figure \ref{fig:mva_cc_res}, these two trainings are shown and no significant difference is seen. 
Therefore the standard, non-resonant training will be used also for the resonant analysis. 




