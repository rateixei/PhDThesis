\chapter{The CMS Electromagnetic Calorimeter}

The CMS ECAL is a high-resolution, hermetic, and homogeneous electromagnetic calorimeter made of 75,848 scintillating lead tungstate crystals divided among a barrel ($|\eta| < 1.48$) and two endcaps ($1.48 < |\eta| < 3.0$) \cite{ECALTDR}.
These crystals are characterized by fast light emission ($80\%$ of light emitted in 25 ns), short radiation length ($X_0 = 0.89$ cm) and small Moliere radius ($R_M = 2.10$ cm). The light emitted by these crystals is detected with avalanche photodiodes (APDs) in the barrel and vaccuum photontriodes (VPTs) in the endcaps. The signal readout is performed with two avalanche photodiodes (APDs) per crystal in the barrel, and one vacuum phototriode (VPT) in the endcaps. These characteristics, translated into precise energy and timing resolutions, are an invaluable tool for the CMS physics program.

Completing the CMS electromagnetic calorimeter system is a preshower detector (ES), based on lead absorbers equipped with silicon strip sensors. It is installed in front of the ECAL endcaps, covering the region $1.65 < |\eta| < 2.6$. The fine granularity of the ES strips (2 mm wide) can resolve the signals of high-energy photons from the decays of neutral pions into two photons, when the separation angle between the photons is small, and can determine precisely the position of the electromagnetic deposits.

During the first year of Run II data taking at 13 TeV, the LHC provided a challenging environment, with one bunch crossing every 25 ns and an average of 10 interactions per crossing (pile up). This is expected to be even more challenging in 2016, with up to 40 pile up interactions. In the 2015 data taking period, the CMS ECAL operated with more than $98\%$ of its channels active, and was responsible for less than $7\%$ of CMS downtime during physics runs.


\section{Detector Components}
\subsection{PbWO$_4$ Crystals and In-detector Electronics}
\subsection{Trigger and Data Acquisition Systems}
\section{Electron and Photon Energy Reconstruction}

The photon and electron energy reconstruction based on ECAL energy deposits is based on the formula: $E_{e,\gamma} = \left[ \sum_{i}\left( S_i(t)\times c_{i} \times A_{i}\right)\times G(\eta) + E_{ES} \right]\times F_{e,\gamma}$, with $A_{i}$, $c_i$ and $S_i(t)$ as, respectively, the per individual channel amplitude, intercalibration constant and light monitoring constant, $G(\eta)$ is the ADC to GeV absolute scale, $E_{ES}$ is the energy deposit in the preshower, and $F_{e,\gamma}$ are the cluster corrections (different for photons and electrons). These terms will be detailed in the next sections.

\subsection{Online Reconstruction}

The online reconstruction of ECAL deposits starts with the amplification and digitization of the signal from the photodetectors attached to the crystals. This is performed by a multi-gain preamplifier (MGPA), and a 12 bit ADC running at 40 MHz. Ten consecutive samples are recorded and used to perform the pulse reconstruction and amplitude extraction.

The time spacing between two consecutive samples from the ECAL readout electronics is 25 ns, which is the same time spacing between two colliding bunches in the LHC. This implies that, during the readout of one pulse, another scintillation process might start in the same crystal, compromising the in-time amplitude reconstruction. To mitigate this effect, also called out-of-time (OOT) pile up, a new online pulse reconstruction method (multifit) was developed to replace the Run I method \cite{weights}.

In the multifit method, the pulse shape is reconstructed based on a fit to the time samples, minimizing $\chi^2 = \sum_{i=1}^{10}{\left( \sum_j^MA_jp_{ij} - S_i \right)^2}/{\sigma^2_{S_i}}$. The samples ($S_i$) are fitted with one in-time pulse shape template, plus up to 9  out-of-time templates ($p_{ij}$) times their respective amplitudes ($A_j$). $\sigma_{S_i}$ is noise generated by electronics associated with the crystal readout chain. The OOT templates have the same shape as the in time one, but are shifted in time by multiples of 1 bunch crossing (1 BX = 25 ns), within a range of -5 to +4 BX around the in time signal (BX = 0). The pulse shapes have been measured in early 2015, in special runs in which the LHC delivered isolated bunches (no OOT pile up).

It has been observed in both data and simulation that, with the multifit method, OOT pile up reconstruction is negligible. The energy resolution improvement, with respect to the Run I amplitude reconstruction method, is substantial especially for low $E_{T}$ photons and electrons, given the larger contribution of deposits from pile up to the total energy.


\subsection{Response Monitoring}

Time dependent corrections must be applied to the reconstructed amplitude due to changes in detector response with radiation exposure. These changes in response are due to decreases in crystal transparency and variations in VPT response in endcaps.
%This change in response is a convolution of two effects: change in crystal transparency and change in VPT response in endcaps.

The changes in the crystal transparency is due to ionizing radiation creating color centers in the lead tungstate. While the scintillation process remains intact, the amount of light detected by the photodetectors decreases. This effect is partially mitigated through thermal annealing, causing the transparency to increase in the absence of radiation.

A light monitoring system is used to monitor the overall changes in response in the ECAL \cite{laser}. It consists of a system of lasers (operating at 447 nm, close to the wavelength of peak emission for lead tungstate) that injects light in each ECAL crystal, which is then read by the standard ECAL readout. The difference between input and read laser amplitudes are then used to calculate correction factors. %The change in transparency per crystal ($R/R_0$) is then related to the ratio between reconstructed amplitude and the injected light amplitude ($S/S_0$) through the formula:

%\begin{equation}
%\frac{S}{S_0} = \left(\frac{R}{R_0} \right)^{\alpha},
%\end{equation}
%where $\alpha$ has been measured in beam tests and is $\approx 1.5$. $S/S_0$ is then used as a correction factor to account for the response changes.

The history of response change measurements is summarized in Figure \ref{fig_laser}. The changes are up to $6\%$ in the barrel and  reach up to $30\%$ at $|\eta| \approx 2.5$, the limit of the tracker acceptance. For high $|\eta|$ regions, changes are up to $70\%$. The recovery of the crystal response during the long shutdown period is visible. The response was not fully recovered, however, particularly in the region closest to the beam pipe. The monitoring corrections are validated by comparing isolated electron energy as measured by ECAL ($E$) and momentum as measured by the CMS Tracker ($p$), before and after light monitoring corrections. It is seen that the measured corrections bring stability to energy measurements with ECAL.

\begin{figure}[h]
\centerline{\begin{minipage}{16pc}
%\includegraphics[width=16pc]{figs/laser_history.png}
\caption{\label{fig_laser} History of channel response changes as measured by the light monitoring system.}
\end{minipage}\hspace{2pc}%
%\begin{minipage}{11pc}
%\includegraphics[width=11pc]{figs/ICs.pdf}
%\caption{\label{fig_miscalib} Residual miscalibration for individual intercalibration methods %and for combination.}
%\end{minipage}\hspace{2pc}%
\begin{minipage}{16pc}
%\includegraphics[width=16pc]{figs/EcalEnergyResolutionGolden.png}
\caption{\label{fig_res} ECAL energy resolution measured with $Z\rightarrow ee$ events for low bremsstrahlung electrons in the ECAL barrel.}
\end{minipage}}
\end{figure}

\subsection{Intercalibration}

A relative calibration procedure in all ECAL channels is performed to ensure uniformity across the detector. Different and independent methods are used to calculate intercalibration constants (ICs), which are then combined to achieve the desired precision of $<0.5\%$. The final 2015 version of the ICs have been calculated with the full $2.6$ fb$^{-1}$ dataset recorded by CMS with B=3.8 T. The following methods are the same as in Run I \cite{calibration}.

The $\phi$-symmetry method is based on the expected uniformity of the energy flux along $\phi$ rings (region with fixed $\eta$). The ICs are calculated to correct non-uniformities in this flux. This method was used in 2015 to translate the latest ICs, calculated with the full 2012 dataset, to the 2015 detector conditions. This was done by scaling the 2012 ICs by the ratio between 2015 and 2012 $\phi$-symmetry ICs.
The $\pi^0/\eta$ method consists of measuring the invariant mass of these resonances' decays to two photons and maximizing their resolutions by varying the ICs iteratively. %This method does not utilize the absolute value of the  $\pi^0$ and $\eta$ resonances not to interfere with the absolute scale calibration.
The $E/p$ method employs the same logic as the light monitoring validation method, comparing isolated electron energy and momentum. An iterative method is used to minimize the spread of the $E/p$ distribution.
The combined intercalibration was obtained from the mean of the individual ICs at a fixed value of $\eta$, weighted by their respective precisions. The residual miscalibration of an intercalibration method, which is related to the final method precision, is calculated as the spread of the difference between the method's ICs and the other methods' ICs at a fixed value of $\eta$. 
%This residual miscalibration can be seen in Figure \ref{fig_miscalib} for the ECAL barrel, where it is shown that 
The combination of ICs achieves the desired goal of less than $0.5\%$ precision.% for the central barrel region. 


\subsection{Absolute Calibration}

$Z\rightarrow ee$ events are used both to set the $\eta$ scale and the absolute calibration \cite{calibration}. The first is developed to ensure that  different $\eta$ regions have the same relative response, while the second (done separately for barrel and endcaps) sets the absolute energy scale.

A dedicated calibration was performed with 0 T data to account for differences in shower shapes in the absence of magnetic field. For example, in 0 T there is no bremsstrahlung radiation outside the main electron cluster deposit, improving the reconstructed energy resolution.

In addition, the calibration was validated with high energy photons and electrons.
%being used in analyses searching for high mass resonances decaying in those particles.
The validation was performed by comparing data and Monte Carlo simulations for high energy electrons from $Z\rightarrow ee$. The calibration was found to be stable to $0.5\%$ ($0.7\%$) for electrons up to $p_{T}=150$ GeV in the barrel (endcap). Possible saturation effects were corrected for with a multivariate technique, but those effects were found to be $<2\%$ for photons arriving from resonance masses less than $1.4$ TeV.

\subsection{High Level Calibrations}

The amount of material in front of ECAL, up to $2X_0$ in the barrel outer regions, produces a high rate of bremsstrahlung radiation from electrons and a high probability of photon conversions. To mitigate this effect, a clustering algorithm is used to recombine energy deposits that come from those processes. The cluster energy is corrected via a multivariate technique, separately for photons \cite{photons} and electrons \cite{electrons}. It also aims to correct other effects, such as in time pile up.

\section{ECAL Performance with Run II data}

The ECAL energy resolution is measured using $Z\rightarrow ee$ events, from an unbinned fit with a Breit-Wigner function convoluted with a Gaussian as signal model. Degradation effects come from the amount of material in front of ECAL and cracks between modules. The resolution, as a function of $\eta$, for low bremsstrahlung electrons in the barrel can be seen in Figure \ref{fig_res}.
%The energy resolution achieved with reprocessed data, which includes the latest intercalibration and calibration constants derived with 2015 data, achieves a resolution that is less than 2$\%$ for $Z\rightarrow ee$ low bremsstrahlung electrons in the central barrel region.
The energy resolution achieved using the latest 2015 calibration constants is less than $2\%$ for low bremsstrahlung electrons in the barrel.
%The reprocessed data is especially better performing in the endcaps, when comparing to processed data with 2012 values for intercalibration and calibration. When ported to physics analysis, this energy resolution implies, for example, in a $\sigma_{eff}/M_h \approx 1.5\%$ (from simulation) in the $H\rightarrow\gamma\gamma$ analysis, where $\sigma_{eff}$ is the smallest interval in $M(\gamma\gamma)$ with $68.5\%$ coverage \cite{hgg}.

\subsection{Performance}
\section{The CMS ECAL Barrel Upgrade}
Einstein's paper: \cite{Einstein}
